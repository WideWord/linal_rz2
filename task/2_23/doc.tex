\section{Задача 2.23}
\subsection{Задание:}
а) Выяснить, образует ли множество функций, ограниченных на некотором $ [a, b] $
линейное пространство относительно обычных операций сложения и умножения на число.
\\
б) Выяснить, является ли линейным подпространством множество векторов плоскости, по модулю не превосходящих 1.
\\
в) Выяснить, является ли множество верхних треугольных матриц порядка $ n $ линейным подпространством
в пространстве всех квадратных матриц поряжка $ n $, и если является, то найти его размерность.
\\
г) Выяснить, является ли линейным подпространством множество векторов в $ n $ - мерном пространстве,
сумма координат которых равна 0, и если является, то найти его размерность.
\subsection{Решение:}
а) Выяснить, образует ли множество функций, ограниченных на некотором $ [a, b] $
линейное пространство относительно обычных операций сложения и умножения на число.
\\[1em]
Пусть дано множество функций $ \mathbb{F} $, ограниченных на $ [a, b] $ и $ f(x), g(x) \in \mathbb{F} $.
Зададим операции сложения $ (f + g)(x) = f(x) + g(x) $,
умножения на число $ (\lambda f)(x) = \lambda (f(x)), \lambda \in \mathbb{R} $.
\\
Докажем что $ f(x) + g(x) \in \mathbb{F} $: функции $ f(x) $ и $ g(x) $ ограниченны сверху на $ [a, b] $,
значит существуют точные верхние грани этих функций $ F, G $ соответственно, тогда точная верхняя грань функции $ f(x) + g(x) $
 - $ F + G $, поскольку она существует, функция $ f(x) + g(x) $ ограничена сверху.
Аналогично доказывается что функция $ f(x) + g(x) $ ограничена снизу. Значит $ f(x) + g(x) \in \mathbb{F} $.
\\
Докажем что $ \lambda f(x) \in \mathbb{F} $: $ f(x) $ ограничена сверху, $ \Rightarrow \exists F $ - верхняя грань $ f(x) $.
$ f(x) < F \Rightarrow \lambda f(x) < \lambda F $. Значит $ \lambda f(x) $ ограничена сверху. Аналогично доказывается что
$ \lambda f(x) $ ограничена снизу. Значит $ \lambda f(x) \in \mathbb{F} $.
\\
Проверим аксиомы линейного пространства:
\\[1em]
$
	(f + g)(x) = f(x) + g(x) = g(x) + f(x) = (g + f)(X)\\[1em]
	(f + g)(x) + m(x) = f(x) + g(x) + m(x) = f(x) + (g + m)(x) \\[1em]
	\exists \theta : \theta (x) = 0, (f + \theta)(x) = f(x) + \theta(x) = f(x)\\[1em]
	\exists -f(x) : -f(x) + f(x) = \theta\\
	(-f + f)(x) = -f(x) + f(x) = (-1)f(x) + f(x) = \theta\\[1em]
	\lambda(f + g)(x) = \lambda f(x)  + \lambda g(x)  = (\lambda f + \lambda g)(x)\\[1em]
	(\alpha + \beta)f(x) = \alpha f(x) + \beta f(x) = (\alpha f + \beta f)(x)\\[1em]
	\alpha(\beta f(x)) = (\alpha \beta)f(x) = (\alpha \beta f)(x)\\[1em]
	(1 \cdot f)(x) = 1 \cdot f(x) = f(x)\\[1em]
$
Значит множество $ \mathbb{F} $ - линейное пространство.
\\[2em]
б) Выяснить, является ли линейным подпространством множество векторов плоскости, по модулю не превосходящих 1.
\\[1em]
Пусть $ L $ - множество векторов плоскости, по модулю не превосходящих 1, $ a = (0, 1), \; b = (1, 0) $.
\\
$ a, b \in L $, когда $ a + b \notin L $, значит $ L $ не является линейным пространством.
\\[2em]
в) Выяснить, является ли множество верхних треугольных матриц порядка $ n $ линейным подпространством
в пространстве всех квадратных матриц поряжка $ n $, и если является, то найти его размерность.
\\[1em]
При сложении двух верхних треугольных матриц получается верхняя треугольная матрица.
При умножении верхней треугольной матрицы на число получается верхняя треугольная матрица.
Так как аксиомы выполняются для множества квадратных матриц порядка $ n $, они будут выполнятся
и для верхних треугольных матриц. Размерность подпространства верхних треугольных матриц будет равна
$ \sum \limits_{i=1}^n i $.
\\[2em]
г) Выяснить, является ли линейным подпространством множество векторов в $ n $ - мерном пространстве,
сумма координат которых равна 0, и если является, то найти его размерность.
\\[1em]
При сложении двух таких векторов получается вектор сумма координат которого равна нулю.
При умножении такого вектора на число получается вектор сумма координт котороо тоже равна нулю.
Аксиомы выполняются для всех векторов. Так как одна координата равна сумме других взятой с противоположным знаком, размерность
такого пространства будет равна $ n - 1 $.
