\section{Задача 2.9}
\subsection{Задание:}
Доказать что всякое комплексное $ z \neq -1, |z| = 1 $ представимо в виде $ z = \dfrac{1+it}{1-it}, t \in \mathbb{R} $.
\subsection{Решение:}
$
	\dfrac{1+it}{1-it}
	=
	\dfrac{\sqrt{1+t^2}e^{i\phi}}{\sqrt{1+t^2}e^{-i\phi}}
	=
	e^{2i\phi}
$, где $ \phi = \operatorname{Arg} (1 + it) $
\\
Так как $ |z| = 1 $, $ z $ можно представить как $ e^{i\theta} $, приравняем: $ i\theta = 2i\phi $.
\\
Значит любое комплексное число $ z \neq -1, |z| = 1 $ представимо в виде $ z = \dfrac{1+it}{1-it}, t \in \mathbb{R} $.
\subsection{Выполним проверку в компьютерной среде Wolfram Mathematica:}
\includegraphics[scale=0.6]{task/2_09/screen.png}
\subsection{Вывод:}
Утверждение доказано и прошло проверку в компьютерной среде.
