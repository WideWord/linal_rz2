\section{Задача 2.22}
\subsection{Задание:}
Доказать, что любую матрицу ранга $ r $ можно представить в виде суммы $ r $
матриц ранга 1, но нельзя представить в виде суммы менее чем $ r $ таких матриц.
\subsection{Доказательство:}
Пусть дана матрица $ A \in M(n, m) $, где первые $ r $ строк являются линейно независимыми, а все последующие являются их
линейными комбинациями. Тогда $ A $ можно представить как сумму
$ A_1, A_2, \dots, A_r $, где
\\[1em]
$
	A_i =
	\begin{pmatrix}
		0 & 0 & \cdots & 0 & 0 \\
		\vdots & \vdots & \ddots & \vdots & \vdots \\
		a_{i1} & a_{i2} & \cdots & a_{i(m-1)} & a_{i(m)} \\
		\vdots & \vdots & \ddots & \vdots & \vdots \\
		k_{j1} & k_{j2} & \cdots & k_{j(m-1)} & k_{j(m)} \\
		\vdots & \vdots & \ddots & \vdots & \vdots \\
		k_{n1} & k_{n2} & \cdots & k_{n(m-1)} & k_{n(m)} \\
	\end{pmatrix}
	, k_{jl}
$ - коэффиценты линейных комбинаций линейно зависимых строк.
\\[1em]
Так как ранг суммы матриц может быть больше суммы рангов этих матриц, количество слагаемых
с рангом $ 1 $ не может быть меньше ранга суммы $ A $.
